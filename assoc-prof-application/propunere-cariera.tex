\documentclass[11pt,a4paper]{article}

% character encoding
\usepackage{ucs}
\usepackage[utf8x]{inputenc}
\usepackage[english,romanian]{babel}

% Support for including graphics
\usepackage{graphicx}
\DeclareGraphicsExtensions{.pdf,.png,.jpg}

% adjust the page margins
\usepackage[scale=0.8]{geometry}

% no indent
\usepackage{parskip}

\usepackage{url}

% required when changes are made to page layout lengths
%\AtBeginDocument{\recomputelengths}

\title{Propunere de dezvoltare a carierei universitare}
\author{șl. dr. ing. Răzvan Deaconescu}
\date{\today}

\begin{document}

\maketitle

% o propunere de dezvoltare a carierei universitare a candidatului, atât din
% punct de vedere didactic, in cazul posturilor didactice, cât si din punct de
% vedere al activitaților de cercetare stiințifica; propunerea se redacteaza de
% catre candidat, cuprinde maxim 10 pagini si este unul dintre principalele
% criterii de departajare a candidatilor;

Întreaga activitate profesională pe care am desfășurat-o în ultimii aproape 14 ani s-a desfășurat în jurul \textbf{Departamentului de Calculatoare} din cadrul Universității POLITEHNICA din București. Este mediul în care m-am dezvoltat profesional și la dezvoltarea căruia am contribuit. Și cel în care doresc să continui să activez.

Știind că sunt foarte atașat de aceste mediu, primesc adesea întrebarea ,,De ce ai rămas în facultate? Ce te atrage?''. Poate că cei care întreabă doresc un răspuns obiectiv sau metrici cantitative, dar nu le am. Argumentul principal este \textbf{plăcerea de a fi parte din comunitatea de educație din lume, activând în mediul socio-profesional din Departamentul de Calculatoare, UPB}. Nu am întâlnit până acum un alt mediu care să îmi ofere aceeași plăcere și satisfacție. Probabil pe plan obiectiv, alt mediu mi-ar putea oferi mai multe rezultate, dar nu cred că starea de bine ar fi aceeași. Oamenii alături de care lucrez, studenții pe care îi îndrumăm, posibilitățile de colaborare cu personalități și organizații, impactul pe care îl am sunt argumente pe care nu le cuantific, dar le simt în fiecare zi. Nu îmi pun niciodată întrebarea de ce fac ceea ce fac, mă trezesc și știu că îmi place și \textbf{nu mă văd făcând altceva altundeva}.

La starea de bine contribuie feedback-ul pe care îl percep la nivelul activităților mele. Și din perspectiva mea și din perspectiva celor din jur, ajut la creșterea disciplinelor predate, a calității educației studenților, a profilului de cercetare al facultății noastre, al legăturilor cu cercetători, profesioniști, universități, firme, al organizării activităților în departament. Astfel îmi este alimentată \textbf{energia și entuziasmul de a continua să fac ceea ce fac și de a descoperi noi moduri de contribuție}.

Pentru a prezenta ce îmi propun în următoarea perioada, ca parte a concursului de avansare la postul de conferențiar, o să sumarizez activitățile din ultimii ani. Atât sumarizarea cât și propunerea de dezvoltare a carierei le voi structura pe 4 piloni, pe care îi consider reprezentativi pentru mine și pentru comunitatea de educație, mediul academic și facultate:
\begin{itemize}
  \item \textbf{educație}: Voi prezenta ce urmăresc să fac pentru îndrumarea studenților, pentru creșterea calității educației la nivel local, național, global, pentru modernizarea procesului de educație, pentru privirea educației dincolo de spațiul studenției.
  \item \textbf{cercetare}: Ținând cont de rezultatele ultimilor ani, voi descrie zonele în care, colaborând cu persoane din facultate și din afara facultății, să împingem înainte zone în care am demonstrat competență, să obținem prestigiu/
  \item \textbf{comunitate}: Fără oameni și fără legături suntem limitați în ceea ce facem. Urmăresc să construiesc să fiu parte în continuare parte din comunități formale și informale centrate în jurul educației, domeniului IT.
  \item \textbf{administrativ}: Activitățile de educație, cercetare și comunitate sunt cu atât mai productive cu cât nu trebuie să ne îngrijorăm de aspecte organizatorice, logistice. Îmi doresc să contribui la părți administrative în ideea de degrevare de astfel de activități pentru a mă concentra pe cei trei piloni de mai sus.
\end{itemize}

\section*{Sumar de activitate}

În ultimii ani o bună parte din activitatea mea se bazează pe declanșarea interesului pentru zona securității. În jurul anilor 2011-2012 am început să lucrez în zona securității, întâi rezolvând provocări de securitate (\textit{challenges}) și apoi sintetizând cunoștințele. Pe acest interes au fost două evenimente definitorii pentru rezultatele ulterioare, evenimente ce au avut loc în 2012-2013:
\begin{enumerate}
  \item dezvoltarea materialelor de securitate de tipul CTF folosite apoi la cursuri în facultate, la \textit{Security Summer School}
  \item colaborarea cu Mihai Chiroiu (și de acolo cu Lucas Davi și Ahmad Reza-Sadeghi) în cercetarea de proiecte iOS
\end{enumerate}

\subsection*{Educație}

În plan de educație, am contribuit în anul 2017 la deschiderea unui program de master în zona securității (\textit{Advanced Cybersecurity}), în coordonarea d-lui Nicolae Țăpuș. Am ajutat la definirea programei, la stabilirea direcțiilor de cercetare pentru studenți și la pregătirea dosarului pentru acreditare. Masterul este acum în al doilea an de desfășurare și nivelul său și al studenților este în creștere.

În ultimii ani am ajutat la dezvoltarea cursului de Introducere în organizarea calculatoarelor și limbaje de asamblare. Împreună cu alți colegi, am restructurat și modernizat programa cursului, am consolidat echipa didactică și am corelat programa cu cea a altor materii. În prezent cursul se bucură de o apreciere puternică din partea studenților și este o bază pentru alte cursuri și pentru activități din zona de securitate, precum \textit{Security Summer School}.

M-am ocupat de organizarea și coordonarea echipelor didactice de la mai multe cursuri. Consider că am abilități de interacțiune, comunicare și organizare foarte bune și am ajutat la crearea de comunități în jurul disciplinelor cursului, din care fiecare membru este mândru să facă parte. Pentru aceasta am creat ghiduri de organizare a echipelor și a resurselor, recomandări de dezvoltare și predare a materialelor, de interacțiune și de îndrumare a studenților. Acestea sunt folosite acum în echipele în care sunt implicat.

Educația nu se rezumă doar la nivelul facultății. Am fost implicat în calitate de director educațional în programul Digital Kids (\url{http://digitalkids.ro}), un program de transmitere de competențe digitale pentru copii între 8 și 12-13 ani. Cred că este important să avem implicare dincolo de facultate, voi insista pe acest lucru în secțiunea de propunere de dezvoltare.

Am fost implicat în livrarea de training-uri profesionale pentru adulți pe subiecte din zona sistemelor de operare, administrării sistemelor și securității. Cred că mediul academic poate oferi suportul pentru educație în carieră continuă. Mediul IT din România și din lume are nevoie de competență și de formare continuă și noi putem contribui la acest lucru.

\subsection*{Cercetare}

Corpul PRECIS nou deschis în campusul UPB a reprezentat un câștig imens la nivelul activităților de cercetare și de comunitate în Facultatea de Automatică și Calculatoare. Mă ocup de coordonarea laboratorului de cercetare PR708 în cadrul căruia abordăm subiecte de sisteme de operare, securitate, virtualizare, compilatoare. Este un cadrul socio-profesional de care sunt foarte atașat, bazat pe comunitatea din fostul laborator de cercetare EG208.

O dată cu pornirea colaborării pe proiecte iOS cu Mihai Chiroiu, am început să fim parte din comunitatea globală de cercetători și profesioniști iOS. Astfel că în acest moment suntem parte a unei echipa care a reușit prima inversare completă de profile de securitate iOS. Pe plan de cercetare am ajuns să avem rezultate excelent, ducând numele universității la conferințe de top precum ACM CCS și AsiaCCS. Avem un articol acceptat la IEEE S\&P 2020, probabil cea mai puternică conferință de securitate din lume.

Rezultatele de cercetare s-au bazat și pe o colaborare constantă și fructuoasă cu Lucas Davi și Ahmad-Reza Sadeghi (TU Darmstadt) și mai recent cu Luke Deshotels și Will Enck (NCSU). Colaborarea cu Will Enck este în desfășurare și urmărim să abordăm subiecte de cercetare în continuare.

\subsection*{Comunitate}

În planul comunității extinse, am înființat și coordonat școala de vară \textit{Security Summer School}, o școală de vară de securitate aplicată care a condus la creșterea imaginii domeniului securității în rândul studenților și a interesului pentru cercetare. Școala de vară este acum la ediția a 6-a.

Am continuat implicarea mea în comunitatea open source, în cadrul ROSEdu (Romanian Open Source Education), RLUG și activități precum OpenAir. Sunt un susținător al modelului open source pentru dezvoltare software dar și de conținut și îl consider de nedezlipit de ideea de educație.

Ca parte a activității de cercetare în zona iOS, am create comunitatea \textit{Malus Security} (\url{https://github.com/malus-security}), comunitate formată din cercetători și studenți care cercetează și dezvoltă proiecte de securitate iOS. În spiritul open source, ulterior publicării rezultatelor, facem accesibil (open source) codul sursă.
comunitatea de iOS Security (Malus Security)

În ultimii ani am fost implicat ca membru al juriului și mentor la concursul de proiecte aplicate pentru elevi InfoEducație (\url{https://infoeducatie.ro}). Este un concurs cu o perspectivă necesară, de a crea aplicații concrete, utile. Cred că este esențial să existe astfel de activități în învățământul preuniversitar și că mediul academic trebuie să aibă deschidere la acestea.

O dată pornită implicarea mea în zona de securitate, am ajustat la crearea comunității de competiții de securitate CTF (\textit{Capture the Flag}) în facultate și în țară. Din ce în ce mai mulți studenți au fost implicați și i-am îndrumat și încurajat spre astfel de concursuri. Am coordonat și coordonez concursurile de securitate OWASP CTF și ACS IXIA CTF, destinate studenților.

Colaborarea cu Ixia, începută în 2006, este esențială pentru sprijinul comunităților din facultate și este o formă de interacțiune constructivă permanentă cu mediul IT din România. În plan secund, sunt implicat în colaborări cu Bitdefender, Thales, Amazon, NXP.

Recent am pornit împreună cu Răzvan Rughiniș o colaborare cu Fundația D (\url{https://dlang.org}) în jurul limbajului D. Un membru central al colaborării este Andrei Alexandrescu, o personalitate în lumea limbajelor de programare. În cadrul colaborării ne concentrăm tot pe aspecte de securitate și modul în care limbajul D poate oferi mecanisme de creștere a securității aplicațiilor.

\subsection*{Administrativ}

În zona administrativă, am urmărit organizare cât mai bună a resursele electronice la nivelul Departamentului de Calculatoare, pentru acces cât mai rapid și pentru a avea o memorie instituțională și proceduri de sprijin. M-am ocupat și mă ocup de wiki-ul departamentului (\url{http://wiki.cs.pub.ro}), de directorul Google Drive al departamentului, liste de discuții. Urmăresc să creez ghiduri și să structurez informația pentru a fi cât mai accesibilă colegilor și studenților.

În particular, am creat ghiduri și proceduri pentru întreținerea sălilor de laborator și a părților logistice din procesul didactic. Aspecte precum instalarea sistemelor, verificări periodice, accesul în sală, calendarul sălii, menținerea sălii, achiziții sunt trecute în aceste ghiduri pe care le folosesc pentru instructajul asistenților și colaboratorilor care predau.

Una dintre soluțiile esențiale la nivelul facultății pentru corectarea scalabilă a temelor de casă este vmchecker (\url{https://vmchecker.cs.pub.ro/ui/}), o aplicație web care permite verificarea automată a temelor de casă la disciplinele din facultate, folosind o soluție bazată pe mașini virtuale. În ultimii ani m-am ocupat de administrarea vmchecker și configurarea sa pentru disciplinele care aveau nevoie de această facilitate.

\section*{Propunere de dezvoltare}

Dacă la concursul de șef de lucrări din septembrie 2012 cadrul propunerii de dezvoltare au fost sistemele de operare, acum cadrul este cel al securității, cumva evident ținând cont de ultimele evoluții ale mele profesionale și de profilul postului 38 scos la concurs.  La nivel tehnic, pe cercetare, urmăresc să mă concentrez pe securitatea sistemului și a aplicațiilor.

Prin cadru nu mă refer la direcție, ci la modul în care voi urmări cei 4 piloni mai sus amintiți. Nu exclud, ba dimpotrivă, sunt deschis la direcții de educație și cercetare altele decât al securității, rămânând să adaptez obiectivele și în funcție de evoluția mea și a mediului în care activez.

\subsection*{Educație}

În zona de educație, eu simt o schimbare de paradigmă pentru care trebuie să fim pregătiți. Avansul tehnologic, dezvoltarea Internetului, posibilitatea de comunicare instantă din puncte diferite al globului, acces facil la orice resurse sunt oportunități și provocări pentru educație. Cred că trebuie să fim cât mai deschiși la integrarea unor forme moderne de educație în procesul didactic. Am început să fac asta la disciplinele în care sunt implicat și voi insista în continuare. Conținut video, forme de predare, învățare și evaluare interactivă, accentul pe discuții, accentul pe lucru individual și în echipă sunt direcții pe care urmăresc să le integrez în zona de educație și din facultate.

Cred că avem o comunitate didactică foarte bună și nu trebuie să limităm contribuțiile noastre didactice doar aici. Există trei niveluri de impact educațional: local, național, global. Voi urmări deschiderea resurselor și a ideilor noastre în cadru cât mai larg de comunități de educație și colaborări cât mai strânse. În particular, în zona sistemelor de operare și a securității urmăresc dezvoltarea de material didactic (de toate formele: scris, interactiv, video) care să fie accesibil dincolo de zona facultății. Mai clar, să fi material accesibil, modificabil, personalizabil, internaționalizabil, urmărind paradigma open source / open access. Deja facem asta la disciplina Sisteme de operare 2 și Security of Information System. Am avut parte de o discuție cu Prateek Saxena (National University of Singapore) care folosește materialele noastre didactice (din limba engleză) și a propus să creăm un set de materiale care să fie universal accesibil și să fiecare să își poată construi un suport didactic pe baza lor.

Sunt de părere că facultatea trebuie să aibă ca prim obiectiv un nivel tehnic cât mai bun al studenților. Este prima prioritate. De aceea eu insist pe competențe tehnice excelente la studenții cu care lucrez. Dar nu este îndeajuns și cu atât mai mult într-o lume în continuă schimbare și nevoi noi nu este de ajuns. Voi insista și încuraja ca studenții să participe la activități extracurriculare, în particular activități în patru direcții: concursuri, de lucru în echipă, de lucru la proiect, de competențe alternative (așa zisele \textit{soft skills}). Cred că educația trebuie să producă oameni ,,rotunzi'' și voi investi energie în această direcție.

Pe baza experienței mele și a ghidurilor pe care le-am creat doresc să le fac internaționalizabile, să schimb idei și să fac sesiuni de prezentare / instruire în special pentru colegii din departament. Cred că sunt informații perfectibile și cred că putem crește calitatea actului didactic discutând și prezentând modurile de organizare pe care le folosim.

\subsection*{Cercetare}

În planul cercetării, principalul obiectiv este să am un subiect de cercetare puternic la care să lucrez (într-o exprimare mai informală ,,un proiect al meu''). Rezultatele curente se bazează pe colaborare și în general contribuții la ideile altora. Urmăresc să trasez o direcție de cercetare la care să fiu principalul investigator. Ținând cont de implicarea mea curentă, acest subiect de cercetare va fi probabil în zona garantării de proprietăți de securitate în sistemele din lumea reală.

Urmăresc consolidarea și crearea de legături cu grupuri de cercetare cu prestigiu global precum National University of Singapore, North Carolina State University, Vrije Universiteit Amsterdam, Ecole Politechnique Federale de Lausanne. Rezultatele de până acum sunt dovada avantajelor unor astfel de colaborări și urmăresc să continui pe această direcție.

La nivel local, date fiind și proiectele curente în care suntem implicați, dar și un set de valori comune și viziune apropiată, doresc să colaborez cât mai strâns cu Costin Raiciu și Mihai Chiroiu. Costin Raiciu are rezultate de cercetare excelente în zona sistemelor, iar Mihai Chiroiu este colaboratorul meu pe zona de iOS în ultimii ani. Amândoi sunt implicați în plan de cercetare și doresc să lucrez în continuare alături de ei.

Ca urmare a colaborărilor și a implicării, dezideratul este publicarea periodică la conferințe de top (rang A și rang A*). Am prins gustul acestor conferințe și cred că putem crește împreună prestigiul facultății și universității. Un obiectiv punctual este acela de a valida rezultatele de cercetare în zona de iOS security cu publicații în jurnale de prestigiu, eventual redactarea unei monografii despre securitatea iOS.

\subsection*{Comunitate}

Partea de comunitate doresc să o dezvolt în special în zona mentoratului de colaboratori noi și preluarea de către aceștia a activităților de comunitate de care m-am ocupat. În secundar voi urmări creșterea vizibilității comunităților open source, de securitate, de educație, având colaboratori care să se ocupe de fiecare în parte. Urmăresc să deleg partea de coordonare a echipelor didactice către colegi și colaboratori care au demonstrat că au abilități de organizare și care sunt dornici să ducă lucrurile mai departe. Persoane precum Răzvan Nițu, Costin Carabaș, Sergiu Weisz vor prelua coordonarea acestor echipe, urmând ca eu să mă ocup de consolidarea imaginii comunităților și legături dincolo de zona facultății.

Tot în zona de interacțiune dincolo de zona facultății, urmăresc să avem o legătură cât mai strânsă în sfera educației cu învățământul pre-universitar, cu organizații formale/informale de educație și cu firme care urmăresc dezvoltarea continuă a carierei. Cred că educația nu se oprește la porțile universității și că aceste comunități din care fac și facem parte pot avea relevanță și în alte zone.

Comunitățile largi în care doresc să mă implic sunt cea de educație (de toate felurile), de securitate și open source. Sunt comunități din care deja fac parte și pentru care acum urmăresc o conexiune cât mai largă, ideal la nivel global.

\subsection*{Administrativ}

În plan administrativ, urmăresc de asemenea delegarea activităților de care m-am ocupat din rațiuni de eficiență și pentru a permite altora să ducă lucrurile mai departe. Activități precum gestiunea resurselor electronice, organizarea proiectelor de licență și master, organizarea sălilor de laborator ar urma să fie preluate de colaboratori. Avem în desfășurare o soluție nouă de îmbunătățire a vmchecker, care va fi coordonată de Alexandru Radovici și care va facilita evaluarea automată a temelor de casă.

Rațiunea implicării administrative este degrevarea efortului pe care îl depunem și să ne concentrăm pe cei trei piloni de mai sus. De aceea urmăresc implicare în echipa de informatizare a facultății și în automatizarea proceselor interne (precum raportări, completare de plată cu ora) și să avem acest timp de ,,uzură'' minim.

Pentru acces mai ușor la rezultatele noastre, doresc să mă implic în lansarea unui depozit (\textit{repository}) instituțional de articole. Aici toți colegii și studenții vor putea încărca pentru arhivare și acces facil toate lucrările lor.

\section*{Concluzie}

Urmăresc în perioada următoare dezvoltarea direcțiilor pe care am activat în ultimii ani, direcții declanșate de implicarea mea în comunitatea de securitate. Mă consider în special un membru al comunității educaționale și cred că tot ceea ce facem trebuie să urmărească modul în care ajutăm pe oamenii din jurul nostru să crească. De aceea implicarea mea va fi structurată pe cei 4 piloni: educație, cercetare, comunitate, administrativ.

Sunt parte a unui mediu care crește și în care mă simt extraordinar. Doresc să continui aici, să continuăm să creștem acest mediul, să dezvoltăm educația la nivel global și să fim parte din comunitatea globală de elită în educație și cercetare în calculatoare.

\end{document}
