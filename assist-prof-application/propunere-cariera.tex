\documentclass[11pt,a4paper]{article}

% character encoding
\usepackage{ucs}
\usepackage[utf8x]{inputenc}
\usepackage[english,romanian]{babel}

% Support for including graphics
\usepackage{graphicx}
\DeclareGraphicsExtensions{.pdf,.png,.jpg}

% adjust the page margins
\usepackage[scale=0.8]{geometry}

% no indent
\usepackage{parskip}

% required when changes are made to page layout lengths
%\AtBeginDocument{\recomputelengths}

\title{Propunere de dezvoltare a carierei universitare}
\author{as. dr. ing. Răzvan Deaconescu}
\date{\today}

\begin{document}

\maketitle

% o propunere de dezvoltare a carierei universitare a candidatului, atât din
% punct de vedere didactic, in cazul posturilor didactice, cât si din punct de
% vedere al activitaților de cercetare stiințifica; propunerea se redacteaza de
% catre candidat, cuprinde maxim 10 pagini si este unul dintre principalele
% criterii de departajare a candidatilor;

Întreaga activitate profesională pe care am desfășurat-o în ultimii șase ani
s-a desfășurat în jurul \textbf{Catedrei} (acum Departamentului) \textbf{de
Calculatoare} din cadrul Universității POLITEHNICA din București. Am
participat în activități diverse, de natură didactică, științifică, de
cercetare, de comunitate, de colaborare cu firme din domeniul IT și de
socializare.  Colectivul Departamentului de Calculatoare este unul dinamic,
divers în care îmi regăsesc foarte multe valori și pe care îl văd evoluând din
toate punctele de vedere.

În cadrul acestui colectiv îmi propun sa activez, în continuare, în următorii
ani de zile. Pe parcursul celor șase ani de activitate am dezvoltat și
menținut relații profesionale și de strânsă prietenie cu mulți dintre colegii
de departament. Îmi doresc ca aceste relații să continue și să se dezvolte
pentru un dublu beneficiu: atât al meu cât și al colegilor de departament.

Pe principiul \textbf{continuității} și doresc să îmi construiesc profilul
profesional în următorii ani. Voi rămâne prolific și implicat în activitățile
de până acum, dar îmi propun să dezvolt și să extind aceste activități.
Filonul acestei dezvoltări îl va reprezenta comunitatea din cadrul
Departamentului de Calculatoare: cadre didactice, studenți, colaboratori,
firme. Prin implicare, feedback permanent, transparență, prietenie și suportul
comunității, îmi propun să fiu unul dintre principalii piloni de dezvoltare ai
domeniului Sistemelor de Operare și a domeniilor conexe.

\section*{Sumar de activitate}

În cadrul celor șase ani de activitate, partea didactică am alocat-o
domeniilor Sistemelor de Operare și Rețelisticii, în timp ce partea
științifică am desfășurat-o în contextul Sistemelor Peer-to-Peer, domeniu în
care am lucrat și pentru teza de doctorat. Aceste activități au fost
completate de colaborări cu firme precum Ixia, Orange, VirtualMetrix și
contribuții în comunitatea open source, prin intermediul comunității locale
ROSEdu.

Figura~\ref{fig:activitate} reflectă domeniile de activitate și interes din
această perioadă.

\begin{figure}[h]
  \begin{center}
    \includegraphics[width=0.7\textwidth]{img/activitate}
  \end{center}
  \caption{Activitate până în prezent}
  \label{fig:activitate}
\end{figure}

Domeniul meu de suflet, dezvoltat încă din anul trei din ciclul de licență,
este cel al Sistemelor de Operare. Am avut ocazia să activez ca membru al
echipelor cursurilor de Utilizarea Sistemelor de Operare, Sisteme de Operare
și Sisteme de Operare 2, de multe ori ca lider de echipă. Consider că am
contribuit la construirea unor echipe foarte bine pregătite, entuziaste și cu
abilități didactice deosebite. Acest lucru se reflectă în feedback-ul transmis
de studenți pe parcursul anilor. Cele trei discipline de mai sus reprezintă
coloana vertebrală a componentei didactice pe care îmi propun să o susțin în
perioada următoare.

Pe plan secund, am activat, din punct de vedere didactic, în cadrul echipelor
cursurilor de Rețele Locale și Managementul Proiectelor Software și am
asigurat suport pentru discipline de master din domeniul Sistemelor de Operare
și Rețelisticii. Am avut ocazia să mixez elemente din mai multe domenii și să
dezvolt un subdomeniu de interes de administrare de sistem, prin susținerea
certificărilor LPIC.

Împreună cu echipele cursurilor de mai sus, am folosit resurse electronice
diverse care au facilitat desfășurarea activităților: liste de discuții,
wiki-uri, repository, calendare și cataloage electronice, aplicația vmchecker
pentru corectare automată, aplicații de coordonare (Redmine), facilități
Moodle. Consider că astfel de resurse sunt indispensabile pentru buna
desfășurare a activităților didactice și susțin și voi susține adoptarea lor
și în alte discipline.

Activitățile didactice au fost susținute de publicarea unor cărți, precum
Rețele Locale și Introducere în Sisteme de Operare. Acestea au devenit
manualele cursurilor de Rețele Locale și Utilizarea Sistemelor de Operare.
Experiența obținută în urma dezvoltării de conținut didactic va fi folosită în
viitor în scopuri similare.

Pe planul inovației didactice, am fost unul dintre promotorii, dezvoltatorii
și susținătorii jocului ,,World of USO'', competiție între studenții de anul 1
ajunsă deja la a cincea ediție. Pe lângă câștigurile de cunoaștere și
divertisment aduse studenților, ,,World of USO'' are meritul de a fi ajutat la
construirea comunității de studenți, atrăgând dezvoltatori și contribuitori
din rândul studenților participanți. Este unul dintre principalele proiecte
ale comunității ROSEdu și un model de dezvoltare open source în mediul
academic.

Proiectul P2P-Next și domeniul Sistemelor Peer-to-Peer au asigurat cadrul
dezvoltării mele din punct de vedere științific. Teza de doctorat și
publicațiile pe care le-am realizat sunt centrate pe acest domeniu, cu
infiltrații din Sisteme de Operare și Rețelistică. Deși îmi propun ca domeniu
principal de activitate cel al Sistemelor de Operare, doresc să mențin
interes și contribuții în domeniul Sistemelor Peer-to-Peer, considerându-mă un
fan al conceptului de descentralizare și a provocărilor tehnice oferite de
acest concept.

Activități de cercetare am desfășurat și în cadrul colaborărilor cu studenții
de licență și master, pentru proiecte de diplomă și de cercetare. Grație
colaborării cu Ixia și VirtualMetrix, am coordonat proiecte interesante în
domeniul Sistemelor de Operare, proiecte cu elemente atât științifice cât și
aplicate (dezvoltare unui microkernel, virtualizare, disponibilitate ridicată,
proiecte de virtualizare, proiecte de securitate).

\section*{Propunere de dezvoltare}

După cum am descris mai sus, domeniul Sistemelor de Operare reprezintă pilonul
central pe care mi-l propun în următorii ani de carieră. Acesta va fi susținut
de relații cu grupuri de cercetare cu interese similare din străinătate,
colaborare cu firme de profil și activitate în comunitățile open source. Pe
plan secund îmi doresc să asigur contribuții în domenii precum Rețelistică,
Administrare de Sistem (profesional), Management și Inginerie de Proiecte
Software, Sisteme Peer-to-Peer. Cadrul va fi asigurat de comunități care vor
reuni persoane cu interese similare.

Figura~\ref{fig:viziune} este o transpunere a viziunii de mai sus.

\begin{figure}[h]
  \begin{center}
    \includegraphics[width=0.7\textwidth]{img/viziune}
  \end{center}
  \caption{Viziunea carierei în perioada următoare}
  \label{fig:viziune}
\end{figure}

Activitățile didactice se vor desfășura ca o continuitate a activităților de
până acum, cu integrarea noilor tehnologii apărute și cu extinderea
competițiilor studențești precum ,,World of USO''. Pe baza colaborării cu
entități externe, precum comunități open-source, grupuri de cercetare, sau
firme, îmi propun dezvoltarea curriculei didactice și inițierea de proiecte de
cercetare. În jurul acestor activități se vor dezvolta sau vor oferi suport
comunități precum cele din Figura~\ref{fig:viziune}.

Concret, îmi propun următoarele:
\begin{itemize}
  \item scrierea unui suport pentru cursul de Sisteme de Operare, cu
    acoperirea integrală a conceptelor prezentate; obiectivul este crearea
    unei cărți cu popularizare internațională;
  \item unificarea conținutului prezentat în cadrul cursurilor de Sisteme de
    Operare în toate cele trei serii (CA, CB, CC) pe modelul cursului de
    Utilizarea Sistemelor de Operare, obținând o pregătire unitară a
    studenților;
  \item inițierea și susținerea organizării unor conferințe pe tematică de
    educație și open source ca parteneriate între mediul academic, comunități
    open source și firme de profil;
  \item dezvoltarea de legături cu grupurile de Sisteme de Operare din alte
    universități tehnice din țară (Universitatea din București, Universitatea
    Tehnică din Cluj Napoca);
  \item dezvoltarea de relații de colaborare cu grupuri de cercetare de
    profil, conduse de persoane cu reputație precum Herbert Bos, George
    Cândea, Liviu Iftode, Frank Bellosa;
  \item susținerea unei colaborări cât mai bune cu Intel România cu interes
    pe subiecte de sisteme de operare, programare de nivel scăzut și open
    source;
  \item extinderea competițiilor de tipul ,,World of USO'' la alte cursuri
    (precum Sisteme de Operare sau Sisteme de Operare 2);
  \item susținerea contribuțiilor în upstream în proiecte open source, prin
    competiții și concursuri de tipul ,,Upstream Challenge'' (sporsorizat în
    prima ediție de Intel România) sau ,,Ixia Challenge'';
  \item realizarea de proiecte de diplomă și de master în co-tutelă cu firme
    de profil, pe baza experienței colaborării cu Ixia și VirtualMetrix;
  \item inițierea și susținerea de proiecte naționale și internaționale de
    cercetare în aria sistemelor de operare, sistemelor încorporate,
    securității, virtualizării;
  \item implicarea în disciplinele de master legate de sisteme de operare și
    securitate (în particular, direcția SRIC -- \textit{Securitatea Rețelelor
    Informatice Complexe}) și susținerea unor noi direcții în cadrul
    disciplinelor.
\end{itemize}

Activitățile la care îmi propun să particip se vor desfășura pe trei planuri:
planul didactic, planul științific și de cercetare și plaul de competențe
profesionale.

Planul didactic se va reflecta în cursurile și laboratoarele în care voi fi
implicat și interacțiunea cu studenții. Folosirea unor resurse electronice
adecvate, promovarea acestora, susținerea de competiții între studenți,
promovarea feedback-ului și a transparenței sunt elemente centrale ale
planului didactic. Completarea activităților normate în plan didactic o voi
realiza prin susținerea de workshop-uri și școli de vară în domenii de
interes: sisteme de operare, administrare, open source.

Planul de cercetare se va baza pe relațiile stabilite cu alte grupuri de
cercetare de interes și colaborarea la proiecte naționale și internaționale.
Mă voi baza pe profesionalism, pe încredere în colaboratori și pe experiență
în obținerea de rezultate de cercetare. Partea de cercetare se va reflecta și
în proiectele de diplomă și de master în care mă voi implica; ținând cont de
nivelul foarte bun al studenților voi milita activ pentru ca aceștia să își
continue cariera în mediul academic ca masteranzi, cercetători sau doctoranzi.

Planul de competențe profesionale este complementar planului de cercetare.
Dacă în planul de cercetare, unul dintre obiective este atragerea studenților
în carieră de cercetare, în acest plan îmi propun să contribui la o cât mai
bună aliniere a studenților la cerințele pieței. Mă bazez pe un set de relații
foarte bune cu firme precum Ixia, VirtualMetrix, Intel care să ofere
feedback-ul și suport dezvoltării curriculare corespunzătoare. Voi contribui
la adaptarea continuă a curriculei și la stimularea diversității preocupării
studenților.

\section*{Cadrul de construire a carierei}

Cadrul prin care îmi propun construirea carierei se bazează pe un set de
valori: feedback, transparență, deschidere la nou, comunitate. Mă bazez pe
susținerea acestor valori din partea colectivului Departamentului de
Calculatoare și pe promovarea lor în rândul colaboratorilor. Consider că
dezvoltarea domeniului de Sisteme de Operare, a domeniilor conexe, a carierei
mele și a colaboratorilor sunt dependente de respectarea și susținerea acestor
valori.

\textbf{Feedback-ul} continuu este modul prin care sunt evaluat și evaluez;
reprezintă este cadrul unei îmbunătățiri perpetue. Dezvoltarea unei comunități
și a unor relații profesionale eficiente se bazează pe transmiterea și
încorporarea feedback-ului. Voi susține și voi folosi feedback-ul în
activitățile în care sunt implicat, de natură didactică (feedback de la
studenți sau în cadrul echipelor de cursuri), de natură științifică (sesiuni
de revizie, prezentări interne) și de dezvoltare profesională (discuții
libere, consultanță).

\textbf{Transparența} informației și a deciziilor sunt fundamentale pentru
construirea de echipe, grupuri și comunități deschise, implicate și amicale.
Experiența din cadrul echipelor în care am contribuit acordă transparenței
rolul fundamental în buna desfășurare a echipelor și comunicarea eficientă
între membri. Pe lângă avantajele de mai sus, transparența permite o atmosferă
relaxată: toată echipa știe tot ce se întâmplă; dacă există o nemulțumire, va
fi manifestată și folosită ca feedback.

Într-un domeniu atât de dinamic precum cel al calculatorelor și tehnologiei
informației, \textbf{deschiderea la nou} este obligatorie pentru orice
profesionist.  Tehnologiile noi trebuie îmbrățișate și evaluate încă din
clipa apariției.  Într-un mediu precum cel academic, lipsit de presiunea unui
release iminent, deschiderea la nou este un diferențiator puternic față de
comunitățile cu interes mai degrabă comercial. Am fost și îmi propun să rămân
o persoană însetată de cunoaștere, de următoarele mari descoperiri, de soluții
noi la probleme vechi. Îmi propun să am ochii permanent deschiși, mintea la
fel de deschisă și entuziasmul la maxim.

Am susținut construirea și întreținerea mai multor \textbf{comunități} în
jurul Departamentului de Calculatoare. Îmi face plăcere să le refer ca un
,,ansamblu de comunități cu suport reciproc'': comunitatea Systems,
comunitatea ROSEdu, comunitatea cursurilor de Sisteme de Operare, comunitatea
administratorilor, comunitatea Cisco, comunitatea IxLabs, comunitatea VMXL4.
Mă consider o persoană de comunitate, care se simte bine și care se dezvoltă
în cadrul comunităților. Marea parte a activităților pe care le desfășor, fie
de natură profesională, fie de natură socială, au găsit, găsesc și vor găsi
motivație în beneficiile aduse comunităților în care activez: grupuri de
oameni pasionați și implicați, care aderă la un set de valori comune și
lucrează pentru binele colectiv.

În calitate de membru avansat în Departamentul de Calculatoare, îmi propun să
îmi consolidez nivelul de entuziasm și eficiență și să dezvolt domeniul
Sistemelor de Operare și a domeniilor conexe. Pentru acest lucru mă bazez atât
pe abilitățile tehnice și non-tehnice proprii cât și pe aportul comunităților
din care fac și voi face parte. Menținerea și crearea unor relații trainice cu
viitorii membri ai Departamentului, cu entități externe, precum firme, și cu
membri ai comunității academice și ai comunității open source, promovarea
ideilor de comunitate, transparență și feedback, dezvoltarea de conținut
didactic modern și competitiv sunt direcții pe care voi insista în perioada
următoare.

Îmi doresc să construiesc o carieră academică și o reputație profesională
excelente, care să asigure succes și împlinire personală și o vizibilitate
crescută a Departamentului de Calculatoare.

\end{document}
